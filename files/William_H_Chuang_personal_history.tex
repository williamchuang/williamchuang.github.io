\documentclass[11pt]{amsart}
\usepackage{amssymb, amscd, amsmath, amsthm}
\usepackage{hyperref}
\usepackage{graphicx}
%%%%%%%%%%%%% box
\usepackage{tcolorbox}

\newtheorem*{utheorem}{Theorem}
\newtheorem*{ucorollary}{Corollary}
\newtheorem{theorem}{Theorem}
\newtheorem{lemma}{Lemma}
\newtheorem*{ulemma}{Lemma}
\newtheorem{definition}[theorem]{Definition}
\thispagestyle{empty}

\usepackage{hyperref}
\usepackage[margin=0.9in]{geometry}

\begin{document}

\title{Personal History}
\author{William H. Chuang}
\date{\today}
\maketitle


%I have always wanted to pursue a PhD and be a Doctor just like my dad. When I was little, I was the first witness to see him jump out of the chair from piles of papers and books and thrilled to tell my mum of his discoveries. Most of the time, my father is a calm and quiet person. Only for those moments, he seemed excited and energetic. Later I learned those were the aha moments and they are rare and precious. PhD and Doctor is a long journey that requires focus, accumulation, and responsibility. It is not just personal pursuit; it also comes with educational responsibilities.

I have always aspired to pursue a PhD and become a professor just like my father. When I was a kid, I saw him jump out of his chair from mounds of papers and books, thrilled to tell my mother about his discovery. My father is, for most of the time, a calm and quiet individual. He appeared excited and energetic only for those few moments. I guess that is the nascent stage of my aspiration to pursue a PhD, to experience the same joy and excitement just like my dad did. Later on, I realized that those were the aha moments, which are extremely rare and valuable. A doctorate degree is the culmination of a long path that necessitates concentration, accumulation, and accountability. %With the path I have taken in Mathematics and who I am today, I am confident that I am prepared for all of them to be a good candidate for the PhD program at the University of Arizona.



It took some time and a few turns for me to realize Mathematics is something I genuinely enjoy and want to pursue. 

It was the fall of 2007 when I first started college at National Dong Hwa University (NDHU) in Taiwan, and my major was Physics. Within two and a half years, I had completed major required courses and was hired by the Physics Department to teach and assist my colleagues with difficult problems. I took a medical leave of absence from NDHU in the spring of 2010 due to pneumothorax and returned to Taipei to live with my parents for better recovery. After passing the admission exam, which was deemed the most difficult exam at the time with a less than 4$\%$ acceptance rate, I transferred to National Taiwan University (NTU), Taiwan's top-ranked university with the first class hospital, to live with my family and receive better medical care. I had two relapses during 2011 and 2013, and had to take a medical leave of absence from NTU until I was fully recovered. Throughout these medical leaves, I did thousands of problems in calculus and physics. It was also during that time I understood the link between physics and mathematics, and how mathematics could elevate physics study.




It was from Dr. Paul Zeitz’s problem solving course that I developed my interest in proving theorems. It was to find a pattern from the number of regions that can be divided by n lines in two-dimensional Euclidean space. Although I studied math when I was in Taiwan, my focus was on how to apply math to real-life problems. For the first time, I derived a proof independently by myself. I experienced joy and excitement at that specific moment! From there, I started to take more and more pure math courses.

In Fall 2017, I got an opportunity to attend the MASS program at Pennsylvania State University. I participated in three expository research projects. In the meantime, I took MASS honors courses that were beyond my level including algebra (Elliptic Curves and Elliptic Functions), analysis (Functional Analysis), geometry (Knot Theory), seminars, and colloquium.

I learned quite a lot from the faculty of the program, in particular from Prof. Sergei Tabachnikov’s courses where the notions of knot theory, real and complex analysis, Galois theory, differential geometry, symplectic geometry, and special linear groups were introduced. The most exciting moment was towards the end of the semester when he introduced a smallest working example of Kontsevich’s integral to construct the universal Vassiliev invariant of a knot. Each week we were assigned somewhere between five to ten new problems that were original and never being published. It was challenging but surprisingly I did well and I got full marks on almost all those intriguing assignments. When I look back, I really appreciate this program. I know for the first time what it feels like to be a mathematician and I became really fascinated by the modern development of pure math from starting a research from hands-on and original problems.


The first real immersed experience with mathematical research happened when I came back to San Francisco. In the spring of 2018, I had a chance to do an independent study with Prof. Zeitz on Analytic Number Theory by using Jameson’s The Prime Number Theorem and with two references: one is Ahlfors’ Complex Analysis, and the other is The Prime Numbers and their Distribution, G. Tenenbaum and M.M. France. Each week I met Prof. Zeitz once and sometimes twice to discuss my progress on proving Prime Number Theorem with explicit details. I learned how to analyze and understand different views and found a better way to read a book on my own from Prof. Zeitz. Through that summer, I continued and finished a reading project on Serge Lang’s Introduction to Modular Forms and Elliptic Curves Diophantine Analysis with the help of Elias M. Stein’s Complex Analysis.

I enrolled in a graduate program on pure math at San Francisco State University to strengthen my understanding of mathematics, because I would like to gain more exposure in algebra and to learn how to present and teach math in Spring 2019. In the following semester, I had the chance to do independent study with Dr. Joseph Gubeladze. While Dr. Gubeladze demonstrated how he would solve problems and prove theorems, I realized I was capable of thinking it through all by myself even when he tested me with some of his original problems. At the same time, I started my first college teaching experience in the U.S. I learned from Dr. Kim Seashore the responsibility of being a teacher and also how to be a good one. Given the fact I was born and raised in Asia and not an native English speaker, the way to express math ideas for beginners varied significantly. Moreover, it is important to let them discover on their own in group works from Dr. Kim Seashore, Dr. Serken Hosten, and Dr. Eric Hsu.

In the spring of 2020, I took real analysis II with Dr. Alex Schuster. It was a very practical guide as he introduced a systematic way to deal with infinite series following William Wade's book. Later when I was working on my thesis, I realized how useful this tool box was. Not just Learning how to use a tool but also understanding why it works is the most important concept I learned from Dr. Schuster.

The following course of real analysis II recommended by Dr. Schuster is Dr. Emily Clader’s topology course. It was astonishing that everything in topology can be introduced in such a nice way that even for some students who have taken real analysis, they can still understand it. Through her teaching, I learned to write a proof to communicate with others, and it should be a pleasant and enjoyable experience.

In Fall 2020, I had a chance to take Dr. Sheldon Axler’s course in measure theory, and later in Spring 2021, I took the following course in functional analysis. Those courses were based on his new book ``Measure, Integration, $\&$ Analysis''.  This experience gave me a chance to closely observe how an established mathematician constructs everything covered in his book from scratch. I found that I became more and more interested in making conjectures and developing new notions or tools for proving theorems just within these two courses.

Math is about question asking and conjecturing. In the same semester, I had a chance to do an independent study on the estimation of the Hausdorff dimension of the limit set of Kleinian groups with Dr. Chun-Kit Lai. Later it was extended to my thesis topic. I learned how to ask questions and conjectures that may lead to new mathematical results. At the beginning of the first months, the tactics failed again and again to penetrate the heart of the problem which was to directly compute Hausdorff dimension even for the simplest example. However, each failure brought me closer to an Eureka moment that finally led to a way to find the root cause of why my previous tactics failed. By eliminating each root cause, a visual method that can directly compute the Hausdorff dimension of some given Schottky group was developed. Furthermore, during the thesis writing process, I also learned how to thoroughly write a publishable proof by checking various aspects, generating several explicit examples down to the details.


All of them gradually shaped me into who I am today and my interest in Mathematics has grown day by day, inspiring me to pursue a PhD in mathematics and to become qualified to make socially meaningful contributions as a professor at college or university.



\end{document}

